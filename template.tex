\documentclass[a4paper,dvipdfmx]{jsarticle}

\usepackage{amsmath}
% 以下2つはダウンロードの必要なパッケージ
\usepackage{fancyhdr}
\usepackage{lastpage}

\fancypagestyle{foot}
{
\fancyhead{}
\fancyfoot{}
\renewcommand\headrulewidth{0pt}
\fancyfoot[R]{ p.\thepage/\pageref{LastPage}}
}

\makeatletter
\newcommand{\figcaption}[1]{\def\@captype{figure}\caption{#1}}
\newcommand{\tblcaption}[1]{\def\@captype{table}\caption{#1}}

% 5.3.10のようにsubsubsectionまで対応させる場合は
% \arabic{section}.\arabic{subsection}.\arabic{equation}
% のように\arabic{section}. の後に\arabic{subsection}. を追加
\def\thefigure{\thesection.\arabic{figure}}
\@addtoreset{figure}{section}
\renewcommand{\theequation}{\arabic{section}.\arabic{equation}}
\renewcommand{\thetable}{\arabic{section}.\arabic{table}}
\@addtoreset{equation}{section}
\@addtoreset{table}{section}

% 表1.1、図1.2、図1.3、表1.4…のように図と表をまとめて通し番号をつけたいときは
% 以下のコマンドも追加 
% 表1.1,図1.1、図1.2、表1.2…のように図と表を別で通し番号をつけるときは無視
% \let\c@figure\c@table
% \let\p@figure\p@table
% \let\cl@figure\cl@table
% \let\thefigure\thetable

\makeatother

\begin{document}
\pagestyle{foot}

\section{title1}


\section{title2}
	\subsection{subtitle2.1}


	\subsection{subtitle2.2}


\section{title3}
	\subsection{subtitle3.1}
		\subsubsection{subsubtitle3.1.1}


	\subsection{subtitle3.2}
		\subsubsection{subsubtitle3.2.1}


		\subsubsection{subsubtitle3.2.2}

% 途中別途添付のグラフなどを挿入するなどでページ番号を飛ばすときは以下のコマンドを使用
% n=飛ばすページの総数(このテンプレでは3)
% このコマンドを使用する場合,2回コンパイルするとページ番号が正しい表記になる
\newpage
\pagestyle{foot}
\addtocounter{page}{3}

\section{newpagetitle4}
	\subsection{title4.1}

% 章番号をリセットするときは次のコマンドを使用
% \setcounter{リセットしたい部分}{リセットする番号-1}
\setcounter{section}{0}
	\section{resettitle1}


\end{document}